\newcommand{\empt}[2]{$#1^{\langle #2 \rangle}$}
\begin{tikzpicture}[
    % GLOBAL CFG
    font=\sffamily\scriptsize,
    >=LaTeX,
    % Styles
    cell/.style={% For the main box
        rectangle, 
        rounded corners=5mm, 
        draw,
        very thick,
        },
    operator/.style={%For operators like +  and  x
        circle,
        draw,
        inner sep=0pt,
        minimum height =.2cm,
        },
    function/.style={%For functions
        ellipse,
        draw,
        inner sep=1pt
        },
    vector/.style={% For external inputs and outputs
        circle,
        draw,
        line width = .75pt,
        minimum width=1cm,
        inner sep=1pt,
        },
    gate/.style={% For internal inputs
        rectangle,
        draw,
        minimum width=4mm,
        minimum height=3mm,
        inner sep=1pt
        },
    mylabel/.style={% something new that I have learned
        font=\scriptsize\sffamily
        },
    ArrowC1/.style={% Arrows with rounded corners
        rounded corners=.25cm,
        thick,
        },
    ArrowC2/.style={% Arrows with big rounded corners
        rounded corners=.5cm,
        thick,
        },
    ]

    % Start drawing the thing...    
    % Draw the cell: 
    \node [cell, minimum height =4cm, minimum width=6cm] at (0,0){} ;

    % Draw inputs named
    \node [gate] (forgetgate) at (-2,-0.75) {$\sigma$};
    \node [gate] (inputgate) at (-1.5,-0.75) {$\sigma$};
    \node [gate, minimum width=1cm] (cellgate) at (-0.5,-0.75) {tanh};
    \node [gate] (outputgate) at (0.5,-0.75) {$\sigma$};

   % Draw operators
    \node [operator] (mux1) at (-2,1.5) {$\times$};
    \node [operator] (add1) at (-0.5,1.5) {$+$};
    \node [operator] (mux2) at (-0.5,0) {$\times$};
    \node [operator] (mux3) at (1.5,0) {$\times$};
    \node [function] (tanh1) at (1.5,0.75) {tanh};

    % Draw External inputs? named as basis c,h,x
    \node[vector, label={[mylabel]Cell}] (c) at (-4,1.5) {\empt{\c}{t-1}};
    \node[vector, label={[mylabel]Hidden}] (h) at (-4,-1.5) {\empt{\h}{t-1}};
    \node[vector, label={[mylabel]left:Input}] (x) at (-2.5,-3) {\empt{\x}{t}};

    % Draw External outputs? named as basis c2,h2,h22
    \node[vector, label={[mylabel]Cell}] (c2) at (4,1.5) {\empt{\c}{t}};
    \node[vector, label={[mylabel]Hidden}] (h2) at (4,-1.5) {\empt{\h}{t}};
    \node[vector, label={[mylabel]left:Hidden}] (h22) at (2.5,3) {\empt{\h}{t}};

    % Start connecting all with arrows.
    % Intersections and displacements are used. 
    % Drawing arrows    
    \draw [ArrowC1] (c) -- (mux1) -- (add1) -- (c2);

    % Inputs
    \draw [ArrowC2] (h) -| (outputgate);
    \draw [ArrowC1] (h -| forgetgate)++(-0.5,0) -| (forgetgate); 
    \draw [ArrowC1] (h -| inputgate)++(-0.5,0) -| (inputgate);
    \draw [ArrowC1] (h -| cellgate)++(-0.5,0) -| (cellgate);
    \draw [ArrowC1] (x) -- (x |- h) -| (cellgate);

    % Internal
    \draw [->, ArrowC2] (forgetgate) -- node[right, near end]{\empt{\f}{t}} (mux1);
    \draw [->, ArrowC2] (inputgate) |- node{\empt{\i}{t}} (mux2);
    \draw [->, ArrowC2] (cellgate) -- node[right]{\empt{\tilde{\c}}{t}} (mux2);
    \draw [->, ArrowC2] (outputgate) |- node{\empt{\o}{t}} (mux3);
    \draw [->, ArrowC2] (mux2) -- (add1);
    \draw [->, ArrowC1] (add1 -| tanh1)++(-0.5,0) -| (tanh1);
    \draw [->, ArrowC2] (tanh1) -- (mux3);

    % Outputs
    \draw [-, ArrowC2] (mux3) |- (h2);
    \draw (c2 -| h22) ++(0,-0.1) coordinate (i1);
    \draw [-, ArrowC2] (h2 -| h22)++(-0.5,0) -| (i1);
    \draw [-, ArrowC2] (i1)++(0,0.2) -- (h22);
\end{tikzpicture}
