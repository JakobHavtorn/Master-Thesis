%!TEX root = ../Thesis.tex
\RequirePackage[l2tabu,orthodox]{nag} % Old habits die hard

\newcommand{\papersizeswitch}[3]{\ifnum\strcmp{\papersize}{#1}=0#2\else#3\fi}
\papersizeswitch{b5paper}{\def\classfontsize{10pt}}{\def\classfontsize{11pt}}

\documentclass[\classfontsize,\papersize,twoside,showtrims]{memoir}
\RequireXeTeX

\showtrimsoff
\papersizeswitch{b5paper}{
    % Stock and paper layout
    \pagebv
    \setlrmarginsandblock{26mm}{20mm}{*}
    \setulmarginsandblock{35mm}{30mm}{*}
    \setheadfoot{8mm}{10mm}
    \setlength{\headsep}{7mm}
    \setlength{\marginparwidth}{18mm}
    \setlength{\marginparsep}{2mm}
}{
    \papersizeswitch{a4paper}{
        \pageaiv
        \setlength{\trimtop}{0pt}
        \setlength{\trimedge}{\stockwidth}
        \addtolength{\trimedge}{-\paperwidth}
        \settypeblocksize{634pt}{448.13pt}{*}
        \setulmargins{4cm}{*}{*}
        \setlrmargins{*}{*}{0.66}
        \setmarginnotes{17pt}{51pt}{\onelineskip}
        \setheadfoot{\onelineskip}{2\onelineskip}
        \setheaderspaces{*}{2\onelineskip}{*}
    }{
    }
}
\ifnum\strcmp{\showtrims}{true}=0
    % For printing B5 on A4 with trimmarks
    \showtrimson
    \papersizeswitch{b5paper}{\stockaiv}{\stockaiii}
    \setlength{\trimtop}{\stockheight}
    \addtolength{\trimtop}{-\paperheight}
    \setlength{\trimtop}{0.5\trimtop}
    \setlength{\trimedge}{\stockwidth}
    \addtolength{\trimedge}{-\paperwidth}
    \setlength{\trimedge}{0.5\trimedge}
    \trimLmarks
\fi

\checkandfixthelayout                 % Check if errors in paper format!
\sideparmargin{outer}                 % Put sidemargins in outer position

% Filler
\usepackage{lipsum}

% Large environments
\usepackage{microtype}

\makeatletter
\def\MT@is@composite#1#2\relax{%
  \ifx\\#2\\\else
    \expandafter\def\expandafter\MT@char\expandafter{\csname\expandafter
                    \string\csname\MT@encoding\endcsname
                    \MT@detokenize@n{#1}-\MT@detokenize@n{#2}\endcsname}%
    % 3 lines added:
    \ifx\UnicodeEncodingName\@undefined\else
      \expandafter\expandafter\expandafter\MT@is@uni@comp\MT@char\iffontchar\else\fi\relax
    \fi
    \expandafter\expandafter\expandafter\MT@is@letter\MT@char\relax\relax
    \ifnum\MT@char@ < \z@
      \ifMT@xunicode
        \edef\MT@char{\MT@exp@two@c\MT@strip@prefix\meaning\MT@char>\relax}%
          \expandafter\MT@exp@two@c\expandafter\MT@is@charx\expandafter
            \MT@char\MT@charxstring\relax\relax\relax\relax\relax
      \fi
    \fi
  \fi
}
% new:
\def\MT@is@uni@comp#1\iffontchar#2\else#3\fi\relax{%
  \ifx\\#2\\\else\edef\MT@char{\iffontchar#2\fi}\fi
}
\makeatother


\usepackage{preamble/cool}            % Uses local copy with bug fixed for derivatives (https://tex.stackexchange.com/questions/54425/basic-use-of-derivative-with-cool-package-fails-with-missing-endcsname-inserte)
\usepackage{mathtools}
%\usepackage{amsmath}
%\usepackage{amssymb}
\usepackage{amsfonts}
\usepackage{upgreek}
\usepackage{xfrac}
\usepackage{siunitx}                  % Use \SI{1.23e3}{m/s} and \si{ms^-1} for SI units
\usepackage[version=4]{mhchem}        % Chemical notation by \ce{}
\usepackage{nth}
\usepackage{listings}                 % Source code printer for LaTeX
\usepackage{pdfpages}
\usepackage{lscape}                   % Landscape pages with \begin{landscape}
\usepackage{enumerate}
\usepackage{ifthen}
\usepackage{calc}

% Quotes
\newcommand{\chapterquote}[2]{
    \vspace{0.3cm}
    \begin{flushleft}
        \begin{minipage}[c]{0.85\linewidth}
            \textit{``#1"}
            %\textit{{\color{dtured}``}#1{\color{dtured}"}}
        \end{minipage}
    \end{flushleft}
    \begin{flushright}
        \raggedleft - #2\\
        \rule{\widthof{- #2}}{0.4pt}
    \end{flushright}
    \vspace{0.5cm}
}

% Captions
\usepackage[justification=centering]{subcaption}
\usepackage[justification=justified, singlelinecheck=false, labelfont=bf, skip=\baselineskip]{caption}
\DeclareCaptionLabelFormat{bold}{\textbf{(#2)}}
\captionsetup{subrefformat=bold}

% Paragraphs
%\setlength{\parindent}{0pt}
%\setlength{\parskip}{2ex}
%\renewcommand{\familydefault}{\sfdefault}


% Links
\usepackage[hyphens]{url}             % Allow hyphens in URL's
\usepackage[unicode=false,psdextra]{hyperref}                 % References package
\renewcommand{\sectionautorefname}{Section}
\renewcommand{\subsectionautorefname}{Section}
\renewcommand{\subsubsectionautorefname}{Section}
\renewcommand{\chapterautorefname}{Chapter}
\tolerance 1414
\hbadness 1414
\emergencystretch 1.5em
\hfuzz 0.3pt
\widowpenalty=10000
\vfuzz \hfuzz
\raggedbottom

% Graphics and colors
\usepackage{graphicx}                 % Including graphics and using colours
\usepackage{xcolor}                   % Defined more color names
\usepackage{eso-pic}                  % Watermark and other bag
\usepackage{preamble/dtucolors}
\graphicspath{{graphics/}}

% Language
\usepackage{polyglossia}    % multilingual typesetting and appropriate hyphenation
\setdefaultlanguage{english}
\usepackage{csquotes}       % language sensitive quotation facilities
\usepackage{longtable}
\let\printglossary\relax
\let\theglossary\relax
\let\endtheglossary\relax
\usepackage[acronym, nonumberlist]{glossaries}    % Definition of acronyms
\usepackage{glossary-mcols}
\renewcommand*{\glspostdescription}{}
\glstoctrue
\usepackage[intoc]{nomencl}
\usepackage{etoolbox}
%\setglossarysection{\chapter}

% Bibliography (references)
\usepackage[backend=biber, %biber,
            style=numeric,
            %backref=true,
            abbreviate=false,
            dateabbrev=false,
            alldates=long]{biblatex}
%\usepackage{csquotes}

% Floating objects, captions and references
\usepackage[noabbrev,nameinlink,capitalise]{cleveref} % Clever references. Options: "fig. !1!" --> "!Figure 1!"
\hangcaption
%\captionnamefont{\bfseries}
%\subcaptionlabelfont{\bfseries}
%\newsubfloat{figure}
%\newsubfloat{table}
\letcountercounter{figure}{table}     % Consecutive table and figure numbering
\numberwithin{equation}{section}
\captiontitlefinal{.}

% Table of contents (TOC)
\setcounter{tocdepth}{3}              % Depth of table of content
\setcounter{secnumdepth}{2}           % Depth of section numbering
\setcounter{maxsecnumdepth}{3}        % Max depth of section numbering

% MatLab code inclusion (Setup of listings package for this purpose)
\usepackage[numbered,framed]{matlab-prettifier}
\lstset{
  style              = Matlab-editor,
  basicstyle         = \footnotesize,
  escapechar         = ",
  mlshowsectionrules = true,
}
% \usepackage{color}
%\definecolor{mygreen}{RGB}{28,172,0} % color values Red, Green, Blue
%\definecolor{mylilas}{RGB}{170,55,241}
%\lstset{language=Matlab,%
%    %basicstyle=\color{red},
%    breaklines=true,%
%    morekeywords={matlab2tikz, end},
%    keywordstyle=\color{blue},%
%    deletekeywords={and},
%    %morekeywords=[2]{1}, keywordstyle=[2]{\color{black}},
%    %identifierstyle=\color{black},%
%    stringstyle=\color{mylilas},
%    commentstyle=\color{mygreen},%
%    showstringspaces=false,%without this there will be a symbol in the places where there is a space
%    numbers=left,%
%    numberstyle={\tiny \color{black}},% size of the numbers
%    numbersep=9pt, % this defines how far the numbers are from the text
%    %emph=[1]{for,end,break},emphstyle=[1]\color{red}, %some words to emphasise
%    %emph=[2]{word1,word2}, emphstyle=[2]{style},    
%}
%\lstset{language=Java}
%\usepackage[framed,numbered,autolinebreaks,useliterate]{mcode}


% Todos
\usepackage{totcount}                 % For total counting of counters
\def\todoshowing{}
\ifnum\strcmp{\showtodos}{false}=0
    \def\todoshowing{disable}
\fi
\usepackage[colorinlistoftodos,\todoshowing]{todonotes} % Todonotes package for nice todos
\newtotcounter{todocounter}           % Creates counter in todo
\let\oldtodo\todo
\newcommand*{\newtodo}[2][]{\stepcounter{todocounter}\oldtodo[#1]{\thesection~(\thetodocounter)~#2}}
\let\todo\newtodo
\let\oldmissingfigure\missingfigure
\newcommand*{\newmissingfigure}[2][]{\stepcounter{todocounter}\oldmissingfigure[#1]{\thesection~(\thetodocounter)~#2}}
\let\missingfigure\newmissingfigure
\makeatletter
\newcommand*{\mylistoftodos}{% Only show list if there are todos
\if@todonotes@disabled
\else
    \ifnum\totvalue{todocounter}>0
        \markboth{\@todonotes@todolistname}{\@todonotes@todolistname}
        \phantomsection\todototoc
        \listoftodos
    \else
    \fi
\fi
}
\makeatother
\newcommand{\lesstodo}[2][]{\todo[color=green!40,#1]{#2}}
\newcommand{\moretodo}[2][]{\todo[color=red!40,#1]{#2}}

% Chapterstyle
\makeatletter
\makechapterstyle{mychapterstyle}{
    \chapterstyle{default}
    \def\format{\normalfont\sffamily}

    \setlength\beforechapskip{0mm}

    \renewcommand*{\chapnamefont}{\format\LARGE}
    \renewcommand*{\chapnumfont}{\format\fontsize{40}{40}\selectfont}
    \renewcommand*{\chaptitlefont}{\format\fontsize{32}{32}\selectfont}

    \renewcommand*{\printchaptername}{\chapnamefont\MakeUppercase{\@chapapp}}
    \patchcommand{\printchaptername}{\begingroup\color{dtugray}}{\endgroup}
    \renewcommand*{\chapternamenum}{\space\space}
    \patchcommand{\printchapternum}{\begingroup\color{dtured}}{\endgroup}
    \renewcommand*{\printchapternonum}{%
        \vphantom{\printchaptername\chapternamenum\chapnumfont 1}
        \afterchapternum
    }

    \setlength\midchapskip{1ex}

    \renewcommand*{\printchaptertitle}[1]{\raggedleft \chaptitlefont ##1}
    \renewcommand*{\afterchaptertitle}{\vskip0.5\onelineskip \hrule \vskip1.3\onelineskip}

}
\makeatother
\chapterstyle{mychapterstyle}

% Header and footer
\def\hffont{\sffamily\small}
\makepagestyle{myruled}
\makeheadrule{myruled}{\textwidth}{\normalrulethickness}
\makeevenhead{myruled}{\hffont\thepage}{}{\hffont\leftmark}
\makeoddhead{myruled}{\hffont\rightmark}{}{\hffont\thepage}
\makeevenfoot{myruled}{}{}{}
\makeoddfoot{myruled}{}{}{}
\makepsmarks{myruled}{
    \nouppercaseheads
    \createmark{chapter}{both}{shownumber}{}{\space}
    \createmark{section}{right}{shownumber}{}{\space}
    \createplainmark{toc}{both}{\contentsname}
    \createplainmark{lof}{both}{\listfigurename}
    \createplainmark{lot}{both}{\listtablename}
    \createplainmark{bib}{both}{\bibname}
    \createplainmark{index}{both}{\indexname}
    \createplainmark{glossary}{both}{\glossaryname}
}
\pagestyle{myruled}
\copypagestyle{cleared}{myruled}      % When \cleardoublepage, use myruled instead of empty
\makeevenhead{cleared}{\hffont\thepage}{}{} % Remove leftmark on cleared pages

\makeevenfoot{plain}{}{}{}            % No page number on plain even pages (chapter begin)
\makeoddfoot{plain}{}{}{}             % No page number on plain odd pages (chapter begin)

% Hypersetup
\hypersetup{
    pdfauthor={\thesisauthor{}},
    pdftitle={\thesistitle{}},
    pdfsubject={\thesissubtitle{}},
    pdfdisplaydoctitle,
    bookmarksnumbered=true,
    bookmarksopen,
    breaklinks,
    linktoc=all,
    plainpages=false,
    unicode=true,
    colorlinks=false,
    citebordercolor=dtured,           % color of links to bibliography
    filebordercolor=dtured,           % color of file links
    linkbordercolor=dtured,           % color of internal links (change box color with linkbordercolor)
    urlbordercolor=s13,               % color of external links
    hidelinks,                        % Do not show boxes or colored links.
}
% Hack to make right pdfbookmark link. The normal behavior links just below the chapter title. This hack put the link just above the chapter like any other normal use of \chapter.
% Another solution can be found in http://tex.stackexchange.com/questions/59359/certain-hyperlinks-memoirhyperref-placed-too-low
\makeatletter
\renewcommand{\@memb@bchap}{%
  \ifnobibintoc\else
    \phantomsection
    \addcontentsline{toc}{chapter}{\bibname}%
  \fi
  \chapter*{\bibname}%
  \bibmark
  \prebibhook
}
\let\oldtableofcontents\tableofcontents
\newcommand{\newtableofcontents}{
    \@ifstar{\oldtableofcontents*}{
        \phantomsection\addcontentsline{toc}{chapter}{\contentsname}\oldtableofcontents*}}
\let\tableofcontents\newtableofcontents
\makeatother

% Confidential
\newcommand{\confidentialbox}[1]{
    \put(0,0){\parbox[b][\paperheight]{\paperwidth}{
        \begin{vplace}
            \centering
            \scalebox{1.3}{
                \begin{tikzpicture}
                    \node[very thick,draw=red!#1,color=red!#1,
                          rounded corners=2pt,inner sep=8pt,rotate=-20]
                          {\sffamily \HUGE \MakeUppercase{Confidential}};
                \end{tikzpicture}
            }
        \end{vplace}
    }}
}

% Prefrontmatter
\newcommand{\prefrontmatter}{
    \pagenumbering{alph}
    \ifnum\strcmp{\confidential}{true}=0
        \AddToShipoutPictureBG{\confidentialbox{10}}   % 10% classified box in background on each page
        \AddToShipoutPictureFG*{\confidentialbox{100}} % 100% classified box in foreground on first page
    \fi
}

% DTU frieze
\newcommand{\frieze}{%
    \AddToShipoutPicture*{
        \put(0,0){
            \parbox[b][\paperheight]{\paperwidth}{%
                \includegraphics[trim=130mm 0 0 0,width=0.9\textwidth]{DTU-frise-SH-15}
                \vspace*{2.5cm}
            }
        }
    }
}

% This is a double sided book. If there is a last empty page lets use it for some fun e.g. the frieze.
% NB: For a fully functional hack the \clearpage used in \include does some odd thinks with the sequence numbering. Thefore use \input instead of \include at the end of the book. If bibliography is used at last everything should be ok.
\makeatletter
% Adjust so gatherings is allowed for single sheets too! (hacking functions in memoir.dtx)
\patchcmd{\leavespergathering}{\ifnum\@memcnta<\tw@}{\ifnum\@memcnta<\@ne}{
    \leavespergathering{1}
    % Insert the frieze
    \patchcmd{\@memensuresigpages}{\repeat}{\repeat\frieze}{}{}
}{}
\makeatother
